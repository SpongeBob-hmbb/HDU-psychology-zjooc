\documentclass{article}
\usepackage{ifthen}
\usepackage{array}
\usepackage{ctex}
\usepackage{xeCJK}
\usepackage{amsmath}
\usepackage{esint}
\usepackage{amssymb}
\usepackage{graphicx}
\usepackage{float}
\usepackage{subfigure}
%                    *********************
%                     **** 四项选择题 ****
%                    *********************
%      用法: \choice{ }{ }{ }{ }
 
\newcommand{\fourch}[4]{%~\hfill(\qquad)\\
\begin{tabular}{*{4}{@{}p{0.25\textwidth}}}(A)~#1 & (B)~#2 & (C)~#3 & (D)~#4\end{tabular}}
\newcommand{\twoch}[4]{%~\hfill(\qquad)\\
\begin{tabular}{*{2}{@{}p{0.5\textwidth}}}(A)~#1 & (B)~#2\end{tabular}\\\begin{tabular}{*{2}{@{}p{0.5\textwidth}}}(C)~#3 & (D)~#4\end{tabular}}
\newcommand{\onech}[4]{%~\hfill(\qquad)\\
(A)~#1 \\ (B)~#2 \\ (C)~#3 \\ (D)~#4}
 
\newlength\widthcha
\newlength\widthchb
\newlength\widthchc
\newlength\widthchd
\newlength\widthch
\newlength\tabmaxwidth
\setlength\tabmaxwidth{1\textwidth}
\newlength\fourthtabwidth
\setlength\fourthtabwidth{0.25\textwidth}
\newlength\halftabwidth
\setlength\halftabwidth{0.5\textwidth}
 
\newcommand{\choice}[4]{\settowidth\widthcha{AM.#1}\setlength{\widthch}{\widthcha}
    \settowidth\widthchb{BM.#2}
    \ifthenelse{\widthch<\widthchb}{\setlength{\widthch}{\widthchb}}{}
    \settowidth\widthchb{CM.#3}
    \ifthenelse{\widthch<\widthchb}{\setlength{\widthch}{\widthchb}}{}
    \settowidth\widthchb{DM.#4}
    \ifthenelse{\widthch<\widthchb}{\setlength{\widthch}{\widthchb}}{}
    \ifthenelse{\widthch<\fourthtabwidth}{\fourch{#1}{#2}{#3}{#4}}
    {\ifthenelse{\widthch<\halftabwidth\and\widthch>\fourthtabwidth}{\twoch{#1}{#2}{#3}{#4}}
        {\onech{#1}{#2}{#3}{#4}}}}
    
%%%%%%%%%%%%%%%%%%%%%%%%%%%%%%%%%%%%%%%%%%%%%%%%%%
\title{大学生心理健康教育 \\ {第五讲}}
\author{派大星}
\date{\today}
\begin{document}
		\maketitle
	\subsection*{单选题}
	\begin{enumerate}
		\item 在恋爱关系中怀疑对方不够喜欢自己,经常搜集或不停发现支持这一猜想的细节,这种现象属于(B)
		
		\choice{避免损失}{证实偏见}{心理定势}{事后聪明}
			
		\item 心理学家艾宾浩斯提出的遗忘曲线,告诉我们的记忆规律是(A)

		\choice{记忆内容的遗忘进程是先快后慢的}{记忆内容的遗忘进程是先慢后快的}{记忆的遗忘进程是加速的}{记忆内容的一望进程是匀速的}
		
		\item 小C正在和好朋友参加周末聚会,此时却收到来自学校社团的短信,因为工作原因让他下午一点回电。小C感到很不爽,一直拖着不回电。虽然下午一点的时候他有时间,却在下午三点的时候才回电话。小C拖延回电最有可能是什么原因(D)
		
		\choice{对失败的恐惧}{对成功的恐惧}{对亲近与疏远的恐惧}{想要获得控制权}
		
		\item 心理活动或意识活动对一定对象的指向和集中是(C)
		
		\choice{认知}{意志}{注意}{想象}
		
		\item 通过消除愉快刺激来降低反应频率(如减少零花钱),这属于(D)
		
		\choice{正强化}{负强化}{正惩罚}{负惩罚}
		
		\item 人们试图用过去解决类似问题的方法、策略和规则来解决新问题的一种倾向被称为(C)
		
		\choice{证实偏见}{避免损失}{心理定势}{事后聪明}
		
		\item 关于操作性条件作用理论,下列说法错误的是(B)
		
		\choice{强化是增强反应频次,而惩罚是减少反应频次}{部分强化的不同类型中定比率强化是最有效的}{正强化是通过在某行为反应后呈现愉悦刺激以强化该反应}{负强化是通过在某行为反应后减少厌恶刺激以强化该反应}

	\end{enumerate}
	\subsection*{多选题}
	\begin{enumerate}
		
		\item 下列哪些方法有助于我们战胜拖延行为(ABCD)

		\choice{时间管理}{运动锻炼}{明确目标}{同伴支持}
	
		\item 下列哪些方法可以减少遗忘(ABCD)
	
		\choice{学会劳逸结合}{及时并有效复习}{深度加工所学内容}{合理安排学习内容}
	
		\item 下列哪些说辞可能属于事后聪明(ABCD)

		\choice{事情不是明摆着吗}{早知道他们就要输了}{就料到会发生这样的事}{这只不过是常识而已}
	
		\item 我们可以尝试使用强化和惩罚的方法,逐步减少或改变自己的不良行为,例如(ABD)
	
		\choice{远离强化物}{打破反应链}{经常批评自己}{自我惩罚合同}
	
		\item 下列哪些属于拖延症的产生原因(ABCD)
	
		\choice{当一个人觉得自己没心情做这件事时}{当一个人感到无法完全依靠自己做事时}{当一个人过分追求完美,对自己期望值过高时}{当一个人觉得他人在接近自己,挤压自己或有求于自己时}

	\end{enumerate}
	\subsection*{判断题}
	\begin{enumerate}
		\item 拖延症属于严重心理问题的一种,是指非必要、后果有害的推迟行为\hfill 错误
		
		\item 动物也会学习,与人的学习在本质上并无差异\hfill 错误
		
		\item 大学阶段的学习考虑更多的是掌握专业知识与能力,培养自身的全面素质\hfill 正确

		\item 当惩罚用于控制行为时,其中一个负面影响就是:被施加惩罚的个体可能会习得性无助\hfill 正确

		\item 我们的认知资源容量是有限的,所以人不可能同时完成两项任务\hfill 错误

		\item 经典条件反射会应用于广告营销中,例如俊男靓女的照片往往会引起某些性唤醒\hfill 正确

		\item 通常情况下,舒适安全的宿舍环境更容易让注意力集中\hfill 错误

		\item 大学里学到的有些知识可能这辈子都用不上,所以这些知识就不需要非常认真地学\hfill 错误

	\end{enumerate}
\end{document}
