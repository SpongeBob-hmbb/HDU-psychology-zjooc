\documentclass{article}
\usepackage{ifthen}
\usepackage{array}
\usepackage{ctex}
\usepackage{xeCJK}
\usepackage{amsmath}
\usepackage{esint}
\usepackage{amssymb}
\usepackage{graphicx}
\usepackage{float}
\usepackage{subfigure}
%                    *********************
%                     **** 四项选择题 ****
%                    *********************
%      用法: \choice{ }{ }{ }{ }
 
\newcommand{\fourch}[4]{%~\hfill(\qquad)\\
\begin{tabular}{*{4}{@{}p{0.25\textwidth}}}(A)~#1 & (B)~#2 & (C)~#3 & (D)~#4\end{tabular}}
\newcommand{\twoch}[4]{%~\hfill(\qquad)\\
\begin{tabular}{*{2}{@{}p{0.5\textwidth}}}(A)~#1 & (B)~#2\end{tabular}\\\begin{tabular}{*{2}{@{}p{0.5\textwidth}}}(C)~#3 & (D)~#4\end{tabular}}
\newcommand{\onech}[4]{%~\hfill(\qquad)\\
(A)~#1 \\ (B)~#2 \\ (C)~#3 \\ (D)~#4}
 
\newlength\widthcha
\newlength\widthchb
\newlength\widthchc
\newlength\widthchd
\newlength\widthch
\newlength\tabmaxwidth
\setlength\tabmaxwidth{1\textwidth}
\newlength\fourthtabwidth
\setlength\fourthtabwidth{0.25\textwidth}
\newlength\halftabwidth
\setlength\halftabwidth{0.5\textwidth}
 
\newcommand{\choice}[4]{\settowidth\widthcha{AM.#1}\setlength{\widthch}{\widthcha}
    \settowidth\widthchb{BM.#2}
    \ifthenelse{\widthch<\widthchb}{\setlength{\widthch}{\widthchb}}{}
    \settowidth\widthchb{CM.#3}
    \ifthenelse{\widthch<\widthchb}{\setlength{\widthch}{\widthchb}}{}
    \settowidth\widthchb{DM.#4}
    \ifthenelse{\widthch<\widthchb}{\setlength{\widthch}{\widthchb}}{}
    \ifthenelse{\widthch<\fourthtabwidth}{\fourch{#1}{#2}{#3}{#4}}
    {\ifthenelse{\widthch<\halftabwidth\and\widthch>\fourthtabwidth}{\twoch{#1}{#2}{#3}{#4}}
        {\onech{#1}{#2}{#3}{#4}}}}
    
%%%%%%%%%%%%%%%%%%%%%%%%%%%%%%%%%%%%%%%%%%%%%%%%%%
\title{大学生心理健康教育 \\ {第三讲}}
\author{派大星}
\date{\today}
\begin{document}
		\maketitle
	\subsection*{单选题}
	\begin{enumerate}
		\item 北方人厚道,南方人精明,属于(B)
		
		\choice{首因效应}{刻板效应}{光环效应}{投射效应}
			
		\item 首因效应和近因效应表明,对(C)印象的形成有重要影响。

		\choice{信息内容}{信息的数量}{信息顺序}{具体的真实性}
		
		\item 以下哪条不属于人际交往的心理原则(B)
		
		\choice{互惠效应}{交易原则}{情境控制原则}{自我价值保护原则}
		
		\item 关于孤独心理,下列说法不正确的是(D)
		
		\choice{孤独是很多人都曾有过的消极体验}{孤独给予个体深入认识自我的机会}{孤独带来的不是更多的来源与制与恐惧}{孤独与孤单、孤僻、独处在本质上是相同的}
	\end{enumerate}
	\subsection*{多选题}
	\begin{enumerate}
		\item 在人际交往中最简单的付出是对他人的称赞,下列说法正确的是(ABCD)
	
		\choice{过于频繁的称赞会降低其价值}{真正的语气和表情会增加称赞的可信度}{称赞具体些好,而不是笼统的说很不错}{与及时的恭维相比,人们更加看重事后的回顾}
	
		\item 美国人类学教授霍尔将人际距离划分为哪几种类型?(ABD)

		\choice{亲密距离}{个人距离}{安全距离}{公共距离}
	
		\item 嫉妒产生的原因包括(BCD)
	
		\choice{性格内向}{自私狭隘}{斤斤计较}{虚荣心过强}
	
		\item 在人际交往中,怎样认可别人使对方觉得自己重要(ABCD)

		\choice{倾听他们}{赞许和欣赏他们}{尽可能经常使用他们的姓名和照片}{关注团体的每一个人,而不只是领导或发言人}
	
		\item 下列哪些现象是接近因素在人际吸引中起作用的写照(ABC)
	
		\choice{青梅竹马}{远亲不如近邻}{近水楼台先得月}{酒逢知己千杯少}
	
		\item 下列哪些表现属于不恰当的交谈方式(ABCD)
	
		\choice{连续发问,以致他人难以应付}{随便打断他人讲话,扰乱人家的思路}{老是将话题转移到自己感兴趣的方面去}{因为自己注意力不集中,致使他人重复谈过的话题}
	
		\item 下列哪些做法可以给人留下良好的第一印象(ABCD)
	
		\choice{微笑}{真诚的表达自己的想法}{聊聊符合别人兴趣的话题}{做一个耐心的听众,鼓励别人谈自己}
	
		\item 下列哪些做法有助于建设性处理人际冲突(ABCD)
	
		\choice{对他人尊重}{贴切地表达真实的想法}{控制自己的冲动,设身处地的给予对方反馈}{关注并表达自己情绪情感}
		
		\item 在人际交往中恰当的做法是(ABCD)
		
		\choice{勇于面对自己的不完美}{尊重人与人之间的差异性}{使用第一人称,“我”,“我们”表达观点}{不要直接怪罪指责和抱怨别人}

	\end{enumerate}
	\subsection*{判断题}
	\begin{enumerate}
		\item 在人际交往中,个性吸引力和能力吸引力很重要,外貌吸引力可有可无\hfill 错误
		
		\item 再好的朋友之间也要适当的保持距离\hfill 正确
		
		\item 自我暴露是指把有关自己个人的信息全部告诉别人,与他人共享自己的感受和信念\hfill 错误

		\item 美国社会学家霍曼斯认为人际交往的本质是一种社会交往过程\hfill 正确

		\item 在倾听他人时,为了表达关注,我们需要紧紧地盯住对方的双眼不放\hfill 错误

		\item 大学期间是发展亲密关系的重要时期,需要多与他人交往而尽量避免独处\hfill 错误

		\item 一个人在人际交往中表现的太过完美,可能会适得其反,不受欢迎\hfill 正确


	\end{enumerate}
\end{document}
